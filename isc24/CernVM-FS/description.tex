
%\comment{Upload detailed description of the tutorial content (2.5 pages maximum)
%- Overview and goals of the tutorial (takeaways for the audience)\\
%- Detailed outline of the tutorial\\
%- Description of the hands-on if applicable\\
%- URLs to sample slides and other material\\
%Upload Resume or CV of each presenter, including a list of the recently taught courses or tutorials (2 pages maximum per presenter). Please upload one pdf file that includes all presenters CVs.}

\subsection*{Overview and hands-on aspects}

This tutorial is intended for people with a background or interest in using a uniform software stack on
High-Performance Computing (HPC) infrastructure, % (system administrators, support team members, end users, etc.),
and who are new to CernVM-FS\footnote{\href{https://cernvm.cern.ch/fs}{https://cernvm.cern.ch/fs}}.
No specific prior knowledge or experience with CernVM-FS is required.
We expect this tutorial to be most valuable to people who are interested in using or providing access to one or more existing
CernVM-FS repositories that provide a common software stack that can be used on HPC systems (and beyond).

\paragraph{} 
The tutorial will start by introducing CernVM-FS, explaining what it is exactly, and what it is mainly used for
(world-wide distribution of software installations). The main features that CernVM-FS provides will be covered in detail,
specifically in the context of HPC, including:
\begin{itemize}
\item Automatic and transparent on-demand downloading of repository contents;
\item Automatic updating of repository contents as new software installations get added;
\item Multi-level caching, to improve start-up performance of software;
\item De-duplication of files and compression of data, which help with reducing the required storage space and network traffic;
\item Automatic verification of the integrity of repository contents that is downloaded on-demand;
\end{itemize}

The European Environment for Scientific Software Installations
(EESSI)\footnote{\href{https://eessi.io}{https://eessi.io}}
will be briefly presented as an example CernVM-FS repository.
It will be used throughout the tutorial to demonstrate the configuration, usage, and features of CernVM-FS
during the hands-on exercises and demos.

\paragraph{} 
After this introductory part of the tutorial, participants will be invited to install and configure a CernVM-FS
client system themselves
to get access to EESSI. This will be done on a prepared dedicated virtual machine (VM) in the cloud that can
be accessed via SSH,
on which the necessary (administrator) permissions are available to participants. The primary goal of this is to provide a hands-on experience
with CernVM-FS, which will help significantly to grasp the concepts and features covered in the introductory part.

Subsequently, we will show how to advance towards a more robust and production-ready CernVM-FS setup,
by also explaining and demonstrating how to set up a mirror and proxy server for a CernVM-FS repository.
Prepared installations of both a mirror and proxy server will be available, so participants can re-configure
their CernVM-FS client system to employ them.

\paragraph{}
In the 2nd half of the tutorial, we will focus on more specific aspects of CernVM-FS, including:
\begin{itemize}
\item Configuration of CernVM-FS on HPC infrastructure, which covers how to use CernVM-FS on diskless or offline systems,
      and how to access CernVM-FS repositories if CernVM-FS is not installed system-wide;
\item Troubleshooting techniques, in case accessing a CernVM-FS repository does not work (anymore), or in case
      large latencies are observed when using software provided via a CernVM-FS repository;
\item Performance aspects, primarily evaluating start-up performance of software, and how to mitigate slow software
      start-up;
\item Using containers technologies like Apptainer together with CernVM-FS;
\end{itemize}

Each of these aspects will be presented by means of examples and hands-on demos.
Participants will be able to follow along in the prepared hands-on environment should they prefer doing so,
but in the interest of time we will not reserve dedicated time for
hands-on exercises during the 2nd part of the tutorial. However, the provided presentation material
will be sufficiently detailed so participants can work through the hands-on demos themselves after the tutorial.

We will briefly explain how to get started from scratch with CernVM-FS towards the end of the tutorial by presenting
how to create a CernVM-FS repository and how to set up the necessary infrastructure to make it available,
but this is not the primary goal of this tutorial.

\subsection*{Motivation \& Goals}

With this tutorial we want to introduce the participants to CernVM-FS in the context of HPC,
and show how it is particularly well suited for providing a central software stack for HPC systems.
In addition, we will demonstrate how it enables HPC support teams, software developers, and end users to
collaborate on a shared software stack that can be used on a wide variety of systems, with EESSI as a prime example.

We strongly believe that the knowledge being disseminated through this tutorial will be beneficial
to a diverse group of conference attendees, ranging from experienced HPC system administrators to researchers
who are just getting started with HPC, and additional profiles in between.

%\paragraph{Content Level}
%60\% introductory, 30\% intermediate, 10\% advanced

%\paragraph{Prerequisites}

%Basic knowledge of using a Unix-style command line interface is desirable. Some familiarity with the issues that arise when compiling/installing scientific software and using environment modules is helpful, though not strictly necessary. A virtual machine image or access to a prepared environment on AWS (Amazon Web Services) will be provided for the hands-on, but no prior knowledge is required to use it.

%\paragraph{Tutorial content}
%We will start of by sketching the traditional way in which HPC support teams deal with installing software, and outline the problems with it. Based on these observations, we will motivate the need for tools that support automation and offer a simple yet powerful user interface to efficiently tackle these problems, and explain how EasyBuild and Lmod fulfill these requirements.

% The tutorial covers installing and configuring EasyBuild, and basic usage through hands-on demos and guided examples. It showcases the main features and capabilities, and also briefly describes an alternative hierarchical environment modules organisation supported by EasyBuild and its benefits and implications.
% %The installation and configuration of EasyBuild will be covered, and basic usage will be presented through demos and guided examples. The main features and capabilities will be outlined and explained, and we will briefly describe an alternative approach to organising environment module files in a hierarchical layout, present the benefits and implications of this approach, and show how EasyBuild supports this.
% Next we cover how to add support for new software (versions) to EasyBuild, how EasyBuild can be customized to adhere to site policies, and how software installations can be performed in distributed manner.
% We also compare EasyBuild with similar projects to help attendees decide which is the best approach for them.
% %The second part of the tutorial will cover some advanced topics including adding support to EasyBuild for building and \emph{TODO MISSING PART}
% %In this part we also compare with some competing projects  so attendees can decide whether EasyBuild is the best choice for their use case.
% Extensive guided examples help consolidate the theory in real knowledge.
% %Special emphasis will be put on the guided examples, where the lessons can be consolidated in real knowledge.
% Finally, three EasyBuild case studies are presented.

% The tutorial covers motivation, goals and the high-level design of EESSI, as well as the usage through hands-on demos and guided examples. It showcases the main features and capabilities, and also describes the various ways to access EESSI: native, as a production setup in HPC as well as without administrator rights. Next we cover how to different use cases of EESSI, such as portable worksflows, within a CI as well as adding software to EESSI. Troubleshooting access, Testing of the software and Performance evaluation of EESSI will also be covered. Extensive guided examples help consolidate the theory in real knowledge.

%The tutorial covers several critical aspects of EESSI, including its motivation, goals, and high-level design, as demonstrated through hands-on demonstrations and guided illustrations. The main features and functionalities of EESSI are highlighted, as well as the different ways to access the software stack, such as native installation, use in an HPC production setup, or without administrator privileges. The tutorial also covers various use cases for EESSI, including portable workflows, integration into a CI environment, and also adding software to the EESSI. Troubleshooting access, testing the software, evaluating the performance of EESSI and configuring GPU support are also covered. Extensive instructor-led examples are used to put the theoretical knowledge into practice.

% At the \textbf{J\"ulich Supercomputing Centre}, EasyBuild is used on multiple supercomputers and prototypes, including JUWELS and JURECA. We cover the highly customized setup used at JSC, the concept of \emph{build stages} used to structure the module
% system, and tools like EasyBuild hooks, a developer module and user installations, which facilitate collaboration on the software stack by over 30 experts.

% \textbf{LUMI}, a EuroHPC JU pre-exascale system, is an HPE Cray EX system with AMD Instinct GPUs.
% We will outline why EasyBuild was selected for LUMI, and explain how
% the setup is focused on empowering users to create their own project software environment on top of the main stack,
% with the help of a distributed support team. In addition, we will show how EasyBuild
% integrates with the HPE Cray Programming Environment, and cover why and how the LUMI project maintains
% its own versions of toolchains and develops many LUMI-specific EasyBuild build recipes.

%We will conclude by showing how EasyBuild is used in large production setups, and provide pointers on how to get in touch with its (growing) community for getting help or to contribute back. Additionally, competing projects will be briefly introduced and compared to EasyBuild, so the attendees can decide whether EasyBuild is the best choice for their use case.
% KH: OK, fixed \comment{MG: ``with the respective communities'' $\rightarrow$ ``with its (growing) user community''?}

%\paragraph{Expected attendance}
%Previous events related to the tutorial topic we (co-)organized were well attended. The \emph{``Getting Scientific Software Installed"} Birds-of-a-Feather sessions at SC'13, ISC'14 and SC'14 were well attended, with 74, 40, and over 100~attendees, respectively. The 1st HPC User Support Tools (half-day) workshop we co-organized at SC'14, where a paper on EasyBuild and Lmod entitled \emph{``Modern Scientific Software Management Using EasyBuild and Lmod"} was presented, was attended by about 60 people. The recent 2nd EasyBuild User Meeting had 35 attendees as well as remote participants in the US, Canada, Europe and Australia {\bf not verified!}.

%Based on this, the relevance of the topic to SC'17 attendees (see above), and the feedback we get from the EasyBuild/Lmod communities and other interested parties, we expect this tutorial to be well attended too. We strongly believe an attendance of 40-50 people is likely.

%\paragraph{Ensuring cohesion}
%We will provide a single cohesive set of slides that covers the entire tutorial. This will be backed up by extensive documentation on the EasyBuild and Lmod documentation pages (\url{http://easybuild.readthedocs.io}, \url{http://lmod.readthedocs.io}). Additionally, a prepared virtual machine image---or access to an AWS instance---will be made available, which will be used during the tutorial for the hands-on sessions.


%\paragraph{Previous talks and tutorials, and update for SC'17}
% EasyBuild has been presented at SC and ISC BoFs (\emph{Getting Scientific Software Installed} in SC'13, SC'14, ISC'14,
% SC'15, SC'18, and SC'19) and Workshops (PyHPC'12, HUST'14, and HUST'16). This tutorial was also presented
% (fully online, half-day) at ISC'21 which was well attended with over 25 participants and in-person (half-day) at ISC'22 with about 20 participants.
% The feedback and attendance has always been positive, which is the primary motivation for this tutorial proposal. The organisers have been actively training user support staff and administrators on EasyBuild in various events in the US and Europe since 2011.

%The organisers have successfully worked together on a number of papers, presentation materials,
%and hands-on tutorials on the topic under consideration several times in recent years.
%A common template
%will be created to guarantee
%will guarantee
%a coherent visual appearance, and using our previous experience, we will provide high-quality content and materials. Moreover, the presenters have experience in organising tutorials in the context of SC and ISC, which helps to understand
%what is expected
%expectations
%and how to structure the material
%to help ISC attendees to get the most out of it.

%\paragraph{Sample slides}

%\begin{itemize}
%\item \url{http://sourceforge.net/projects/lmod/files/Lmod_tutorial_sea_2014.pdf}
%\item \url{http://users.ugent.be/~kehoste/Lmod-intro_20150209.pdf}
%\item \url{http://users.ugent.be/~kehoste/EasyBuild-intro-Basel_20150209.pdf}
%\item \url{http://easybuild.readthedocs.io/en/latest/Writing_easyconfig_files.html}
%\item \url{http://hust16.github.io/presentations/eb-hust16.pdf}
%\end{itemize}

%\comment{MG: According to the RFT, they expect info regarding the target audience, percentage of content split as beginner, intermediate, advanced, and estimated number of attendees (with short explanation if applicable).  However, I have no clue on the latter...}
%\comment{KH: Based on attendance at user meetings (and Spack tutorials), I think an estimated attendance of 50-100 people at ISC'19 is reasonable.
% Spack has about 20-30 attendees at last years ISC I think (but Spack is more US focused), while at SC they had 50-75 attendees I think.}
%\comment{MG: 50-100 is a lot for ISC, I would say. The overall number of tutorial attendees is much lower than at SC, and the audience is also a bit different. I would probably say 20-50 -- and if the room in the end is overfull...}
%\comment{KH: OK for 20-50, let's prove ourselves wrong. ;) }


%KH: So you would go with 20-50? Is that ambitious enough?
%MG: I think so. Attendance is lower than at SC. And more business people...
%KH: I'm OK with 20-50, so let's go with that...
%MG: With my initial comment, I wanted to point out that these sections (target audience, percentage, etc.) are expected as part of the tutorial proposal...  So where should we put it?
%KH: the form ask you to pick 20-50 or 50-80, I don't think it needs to go in the PDF (but we can add it there, sure). After the "Travel support" section?
%MG: Well, at some places it suggests to copy from the proposal ;)
%KH: Let's add a small "Expected attendance" section, mentioning 20-50 + motivation for that number (based on attendance at user meetings (50+), PRACE tutorial ~30 I think in 2017), etc.)

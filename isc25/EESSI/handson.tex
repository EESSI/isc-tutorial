A significant portion of the tutorial is used for guided examples and hands-on,
which are fundamental to exposing the benefits of EESSI and highlight the practical nature of the tutorial.
Before each example session the theoretical concepts are explained through a consistent set of presentation materials.
Guided examples will be carried out in a pre-configured environment on virtual machines in the Cloud that can be
accessed by
attendees from their laptops using SSH. 
The examples cover accessing EESSI in various ways (such as in a production HPC setup or via
containers), the use cases of EESSI, adding support for new software to EESSI, troubleshooting, GPU support, selected
advanced features, and how to contribute to EESSI.
% Each example builds on the previous one and hence also showcases a typical workflow.
%The guided examples will happen in the proximity of a coffee break allowing the attendees to continue
%during the coffee break should they want to and ensuring the maximum benefit from the tutorial.

The tutorial is based on previous online tutorials from December 2023
\href{https://raw.githubusercontent.com/eessi/docs/main/talks/20231205-Introduction-to-EESSI-CASTIEL2/20231205-Introduction-to-EESSI-CASTIEL2.pdf}{"An introduction to EESSI"} ($\sim60$ live attendees, see \href{https://www.youtube.com/watch?v=KAYI9oKFLxA}{YouTube for recording}) and
\href{https://raw.githubusercontent.com/multixscale/cvmfs-tutorial-hpc-best-practices/main/files/Best-Practices-for-CernVM-FS-in-HPC-20231204.pdf}{"Best Practices for CernVM-FS in HPC"} ($\sim130$ live attendees, see \href{https://www.youtube.com/watch?v=L0Mmy7NBXDU}{YouTube for recording}),
but will be updated with new developments in EESSI and the CernVM-FS infrastructure.
A tutorial website will be created and referring to the EESSI documentation on
\href{https://www.eessi.io/docs}{https://www.eessi.io/docs} (similar to \href{https://tutorial.easybuild.io/}{https://tutorial.easybuild.io/}).

It should be noted that though this tutorial has never been presented at ISC, all the authors have extensive
experience in other tutorials and workshops, both inside and outside the ISC conference series.
They are also engaged in other collaborative training efforts within the EESSI and EasyBuild community. Providing coherent and streamlined
tutorial is well within the authors' experience.

% This tutorial was presented in the half-day format at ISC'22
% see \url{https://easybuilders.github.io/easybuild-tutorial/2022-isc22}.
% The authors have extensive experience in other tutorials and workshops,
% and are involved in other collaborative efforts within the EasyBuild community.
% Providing a coherent and streamlined tutorial is well within the authors' experience.


%The guided examples sessions provided in this tutorial are fundamental to exposing the benefits of EasyBuild. The attendees will be incrementally walked through an example which they can choose to repeat it for themselves in a prepared environment (supported by the tutorial team). The set of examples are structured sequentially, building on top of each other to follow the typical workflow for a user. Detail will be gradually incremented, with each example exposing deeper capabilities, ensuring that all the attendees learn the concepts in an organised and clear way.

%The examples will be carried out in pre-configured environments available online and accessed using the attendees laptops via SSH. They will cover installation and configuration of EasyBuild, basic EasyBuild usage, adding support for new software to EasyBuild, and how to contribute to the project. The guided examples will happen mostly in the proximity of the coffee break, allowing scope for attendees to use that time should they want to, ensuring the maximum benefit for the tutorial.

%Half of the tutorial time will be focused on the guided examples, highlighting the practical nature of the tutorial.
%Before each guided example session the theoretical concepts will be explained using a consistent set of slides. The
%tutorial contents will be (loosely) based on the tutorial that was given at the PRACE VI-SEEM 2017 Spring School (see
%\url{http://users.ugent.be/~kehoste/EasyBuild_20170425_PRACE_Spring_School.pdf}, relevant subset uploaded to the
%Linklings submission system), where the presentation material will be updated to focus on EasyBuild, reflect recent
%changes in EasyBuild, include guided examples, and integrate with the EasyBuild documentation hosted at
%\url{https://docs.easybuild.io}.

What if there was a way to avoid having to install a broad range of scientific software from scratch on every HPC
cluster or cloud instance you use or maintain, without compromising on performance?

Installing scientific software for supercomputers is known to be a tedious and time-consuming task. Especially as the
HPC user community becomes more diverse, computational science expands rapidly, the diversity of system architectures
increases the application software stack continues to deepen. Simultaneously, we see a surge in interest in cloud
computing for scientific computing. Delivering optimised software installations and providing access to these
installations in a reliable, user-friendly, and reproducible way is a highly non-trivial task that affects application
developers, HPC user support teams, and the users themselves.

This tutorial aims to address these challenges by providing the attendees with the knowledge to stream optimised
scientific software. For this, the tutorial introduces European Environment for Scientific Software Installations
(\emph{EESSI}), a collaboration between various European HPC sites \& industry partners, with the common goal of
creating a shared repository of scientific software installations that can be used on a variety of systems, regardless
of which flavor/version of Linux distribution or processor architecture is used, or whether it’s a full size HPC
cluster, a cloud environment or a personal workstation.

We cover the usage of EESSI, different ways to accessing EESSI, how to add software to EESSI, and highlight some more 
advanced features. We will also show how to engage with the community and contribute to the project.

Can we avoid having to install a broad range of scientific software from scratch on every HPC cluster or
cloud instance we use, without compromising on performance?

Installing scientific software on supercomputers is typically a tedious, time-consuming task, especially as the
HPC user community becomes more diverse, computational science expands, system architecture diversity increases, the
application software stack deepens, and interest in cloud computing for scientific research surges.
Delivering optimised software installations, and providing access to them reliably and reproducibly,
is a highly non-trivial task impacting application developers, HPC user support, and users themselves.

This tutorial will show attendees how they can stream optimized scientific software installations,
by introducing the European Environment for Scientific Software Installations (\emph{EESSI}), a collaboration
between European HPC sites \& industry partners, with the common goal of creating a public shared repository of scientific
software installations usable on a variety of systems, regardless of Linux flavour or processor architecture, or
whether it's a full size HPC cluster, the cloud, or a desktop.

We cover the usage of EESSI, different access methods, adding software to EESSI, and highlight some advanced features.
We also show how to engage with the community and contribute back.


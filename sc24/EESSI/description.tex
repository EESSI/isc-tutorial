\section*{Tutorial description}

The \textbf{European Environment for Scientific Software Installations
(EESSI)}\footnote{\href{https://eessi.io}{https://eessi.io}} project is a collaborative project
between different partners in the HPC community supported by the MultiXscale EuroHPC Centre of Excellence to build a
common stack of optimised scientific software installations for everything from laptops to big HPC systems and cloud
infrastructures.
The project uses EasyBuild to install software on top of a compatibility layer (to give independence from the underlying
OS), and uses CernVM-FS to
distribute the software installations to client systems.

EESSI is motivated by the observation that the landscape of computational science is changing in
various ways: increasing hardware and user community diversity, and the adoption of public and commercial cloud
infrastructure by research communities.

The critical benefit is that \textbf{EESSI is providing the
installations themselves}, \textit{not} installation recipes.
Environment modules are used as the user interface, and automatic
detection of the target system architecture is done.
This means that the binary files that make up the optimised
scientific software installation that the users require are streamed (via CernVM-FS) to the host
platform as they are accessed.

EESSI is an initiative built on the foundations of the EasyBuild community (which is used by well over
100 HPC sites worldwide (incl. JSC, CSCS, the Digital Research Alliance of Canada, LUMI,\ldots)), and goes one
step further by providing a truly uniform software stack.
It already has a wide spectrum of collaborators despite being a relatively young project.

\subsection*{Tutorial Goals}

EESSI provides a collection of scientific software installations that work across a wide range of
different platforms, including HPC clusters, cloud infrastructure, and personal workstations and laptops, without making
compromises on the performance of that software.

The audience will learn how software installations included in EESSI are optimized for specific generations
of microprocessors by targeting a
variety of instruction set architectures (ISAs). The will learn how to access EESSI in different ways such as native,
in a production HPC setup as well as inside containers.

The hands-on components will showcase the use cases of EESSI, how to add support for new software,
troubleshooting, GPU support, and how they can contribute to EESSI.
% They will also learn that EESSI
% project consists of 3 layers, which are constructed by leveraging various open source software projects:
% \begin{itemize}
% \item The filesystem layer uses CernVM-FS to distribute the EESSI software stack to client systems
% \item The compatibility layer levels the ground across different (versions of) the Linux operating system (OS)
%       of client systems that use the software installations provided by EESSI. It consists of a limited set of libraries
%       and tools that are installed in a non-standard filesystem location (a "prefix"), which were built from source for
%       the supported CPU families using Gentoo Prefix.
% \item The top layer of EESSI is called the software layer, which contains the actual scientific software applications
%       and their dependencies.
%       Building, managing, and optimising the software installations included in the software
%       layer is done using EasyBuild, a well-established software build and installation framework for managing
%       (scientific) software stacks on High-Performance Computing (HPC) systems. Environment modules provides a
%       user-friendly interface to end users of EESSI. The initialisation script that is included in EESSI
%       automatically detects the CPU family and microarchitecture of a client system by leveraging the bash tool
%       \texttt{archdetect}.
% \end{itemize}

\subsection*{Relevance for conference attendees}

Application developers, HPC sites, and end users %around the world
spend significant amounts of time on optimised software installations. Surveys conducted at the
\emph{``Getting Scientific Software Installed''} Birds-of-a-Feather sessions that we (co-)organised at both SC and ISC
reveal that this (still) involves a significant amount of `manual' effort.
In the SC'19 survey,
less than half of the respondents consistently automate software installation,
and only ~25\% automate environment module file generation.
Despite these ubiquitous problems,
there is still inadequate collaboration
between HPC sites to date: less than 30\% of respondents indicated that they
work together with other HPC sites regarding software installation, even in most recent surveys.
Hence, an EESSI tutorial is very relevant to SC'24 attendees as this tool helps relieve these burdens and fosters
collaboration.

\subsection*{Target audience}
This tutorial is intended for
\begin{itemize}
    \item End users who want to empower themselves to use a uniform software stack without compromising on
          performance, on top of what is provided centrally by the HPC support team;
    \item System managers, administrators and user support teams, responsible for operational aspects of HPC systems \&
          concerned about hardware-optimised scientific software installations;
    \item Cloud \& CI users, who want to use a common stack of optimised scientific software;
    \item System manufacturers and integrators interested in state-of-the-art software installation tools, who want to
          leverage the collective expertise incorporated in EESSI and EasyBuild.
\end{itemize}

\subsection*{Content level}
60\% beginner, 40\% intermediate

\subsection*{Audience Prerequisites}
Attendees wishing to participate in the guided examples are expected to use their
own notebook computers with a working SSH client.

Detailed information on how to prepare for this tutorial will be provided
through the tutorial website, similar to
\url{https://tutorial.easybuild.io/2022-isc22/practical_info/}.

\subsection*{Tutorial content}
The tutorial covers several critical aspects of EESSI, including its motivation, goals, and high-level design, as
demonstrated through hands-on demonstrations and guided illustrations. The main features and functionalities of EESSI
are highlighted, as well as the different ways to access the software stack, such as native installation, use in an HPC
production setup, or without administrator privileges. The tutorial also covers various use cases for EESSI, including
portable workflows, integration into a CI environment, adding software to the EESSI, and also configuring GPU support.
Extensive instructor-led examples are used to put the theoretical knowledge into practice.

\subsection*{Cohesion measures}
The organisers have successfully worked together on a number of papers, presentation materials,
and hands-on tutorials on the topic under consideration several times in recent years.
A common template
will guarantee
a coherent visual appearance, and using our previous experience, we will provide high-quality content and materials.
Moreover, the presenters have experience in organising tutorials in the context of SC and ISC, which helps to understand
expectations
and how to structure the material
to help SC attendees to get the most out of it.

\subsection*{Previous editions and potential updates}
The tutorial is based on previous tutorials archived at
\url{https://github.com/EESSI/docs/tree/main/talks/20231205-Introduction-to-EESSI-CASTIEL2},
but will be updated with new developments in EESSI and EasyBuild.
This tutorial will also be made available on that site, and will also refer to the EESSI documentation on
\url{https://www.eessi.io/docs}.
While this tutorial has never been presented at SC, the authors have extensive
experience in other tutorials and workshops, both inside and outside the SC and ISC conference series.
Installing scientific software for supercomputers is known to be a tedious and time-consuming task. Particularly as the 
HPC user community becomes more diverse and computational science expands raplidly, as well as increasing diversity in 
system architectures, continues to deepen the application software stack. On the other hand an increasing interest in 
cloud computing for scientific computing can be observed. Delivering an optimised software installations and providing 
access to these installations in a reliable, user-friendly way is a highly non-trivial task that affects the application 
developer, the HPC user support teams, and the users themselves.

This tutorial aims to address these challenges, by providing the attendees with the knowledge to stream optimised scientific 
software. For this, the tutorial introduces European Environment for Scientific Software Instalations (\emph{EESSI}), a 
collaboration between different European HPC sites & industry partners, with the common goal to set up a shared 
repository of scientific software installations that can be used on a variety of systems, regardless of which 
flavor/version of Linux distribution or processor architecture is used, or whether it’s a full size HPC cluster, a cloud 
environment or a personal workstation.

We cover the usage of EESSI, various ways of accessing EESSI how to add software to EESSI, troubleshooting, performance 
evaluation  and highlight some more advanced features. In addition, we will show how to engage with the community and 
contribute to the project.

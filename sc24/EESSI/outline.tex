\subsection*{Outline}

We believe that a half-day format would be most appropriate for this tutorial, as it would allow ample time for guided
examples and hands-on practice, in addition to providing a comprehensive overview of EESSI.

The tutorial outline for the half-day format is included below.
Parts that include hands-on activity are indicated as such.
Attendees will be invited to follow along with hands-on demos in a prepared environment that they will get access to.

\subsubsection*{Half-day format, 3 hours of tutorial content}

\begin{itemize}[style=standard, labelwidth=1.9cm]


    \item [00:00-00:10] \emph{(30min)} \textbf{Tutorial outline + Practical info}

        To kick off the tutorial, we will briefly outline the contents of the tutorial,
        and provide practical information on accessing and using the prepared environment.

        Attendees will be able to access the prepared environment through a standard internet browser,
        without the need to install any software.

    \item [00:10-00:40] \emph{(30min)} \textbf{Introduction to EESSI}

        We will start with a general introduction to the European Environment for Scientific Software
        Installations (EESSI), which covers the motivation, goals, and high-level design of the project.

        We will also briefly cover the relation to the MultiXscale EuroHPC Centre-of-Excellence,
        which is currently funding the development effort for EESSI.

    \item [00:40-01:10] \emph{(30min)} \textbf{Using EESSI} \emph{(hands-on)}

        In the first hands-on part, we will focus on the user experience of EESSI.
        Attendees will learn how they can start using the rich software stack provided by EESSI.

        We will also briefly cover the different ways of accessing EESSI: through native installation
        of CernVM-FS, via a container, or via the \texttt{cvmfsexec} tool.

        In the prepared environment, a native installation of CernVM-FS that provides access to EESSI
        will be readily available.

    \item [01:10-01:30] \emph{(20min)} \textbf{Use cases (workflows, CI)} \emph{(hands-on)}

        After covering the basic usage of EESSI, we will demonstrate additional use cases that
        are enabled by EESSI, including running workflows, and using EESSI in CI environments.

 \item [01:30-02:00] \emph{(30min coffee break)}

 \item [02:00-02:45] \emph{(45min)} \textbf{Adding software to EESSI} \emph{(hands-on)}

        In this part, we will present and demonstrate the semi-automatic procedure for adding software to EESSI.
        Attendees will be able to follow along with a prepared example by stepping through the procedure in a guided
        way.

 \item [02:45-03:15] \emph{(30min)} \textbf{Beyond the basics: GPU support, test suite, etc.} \emph{(hands-on)}

        To wrap up the technical content of the tutorial, we will briefly cover a number of aspects of EESSI that go beyond the
        basics. This includes a detailed overview of how GPUs are supported in EESSI, and covering the goals and usage
        of the portable test suite that is developed in the scope of the project.

        Attendees will be able to follow along with guided examples.

 \item [03:15-03:30] \emph{(15min)} \textbf{EESSI community + closing remarks + Q\&A}

        To conclude the tutorial, we will briefly present how to get in touch with the EESSI community,
        present concluding remarks, and answer (additional) questions raised by attendees.

\end{itemize}

%What if there was a way to avoid having to install a broad range of scientific software from scratch on every HPC
%cluster or cloud instance you use or maintain, without compromising on performance?

Installing scientific software for supercomputers is known to be a tedious and time-consuming task.
The application software stack continues to deepen, especially as the HPC user community becomes more diverse,
computational science expands rapidly, and the diversity of system architectures increases.
Simultaneously, we see a surge in interest in cloud computing for scientific computing. Delivering optimised
software installations and providing access to these installations in a reliable, user-friendly, and reproducible
way is an increasingly highly non-trivial task that affects software developers, HPC user support teams,
and researchers running scientific workloads on HPC systems.

This tutorial aims to address these challenges by introducing the European Environment for Scientific Software Installations
(EESSI, pronounced as \emph{``easy"}), a collaboration between various European HPC sites \& industry partners. The goal of EESSI is to provide
a shared repository of scientific software installations that can be used on a variety of systems, regardless
of which flavor/version of Linux distribution or processor architecture is used, or whether it is a full size HPC
cluster, a virtual machine in the cloud, or a personal workstation.

We cover the basics of EESSI, different use cases for EESSI, how to add software to EESSI, and highlight some more 
advanced features. We will also show how to engage with the community and contribute to the project.

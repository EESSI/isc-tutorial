Installing scientific software for supercomputers is known to be a tedious and time-consuming task. Increasing HPC user
community diversity, as well as hardware complexity, continues to deepen the application software stack. Delivering
optimised software installations and providing access to these installations in a reliable, user-friendly way is a
highly non-trivial task that affects the application developer, the HPC user support teams, and the users themselves.

This tutorial aims to address this issue, providing the attendees with the knowledge to install and manage software in a clean
and reproducible way. For this, the tutorial introduces \emph{EasyBuild}, a software build and installation framework
implemented in Python. It supports implementing (complex) installation procedures concisely, and is able to fully
autonomously perform optimised software installations.
EasyBuild is available under the open source GPLv2 license,
has a thriving community, and supports more than $2,900$ different (scientific) software packages. It is well integrated with different implementations of environment modules (Lmod, Tcl, C) to generate module files automatically and present them in a consistent way.

We cover the installation and configuration of EasyBuild, the basic usage, installing already supported software, adding support for additional software, and highlight some more advanced features.
In addition, we will show how to engage with the community and contribute to the project,
and how EasyBuild is used and integrated on large-scale HPC systems at the J\"ulich Supercomputing Centre
and in LUMI, one of the EuroHPC JU pre-exascale systems, as well as in the European Environment
for Scientific Software Instalations (EESSI) project.
